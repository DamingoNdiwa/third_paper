\documentclass{article}
\usepackage{graphicx} % Required for inserting images
\usepackage{amsmath}
\usepackage{amsfonts}

\title{Interpretable Wasserstein Impact Measure for prior Impact Assessment.}
\author{Damian Mingo, Jack S. Hale}
\date{June 2023}

\begin{document}


\maketitle

\section{Introduction}
Several measures of prior impact assessment have been developed. These measures of prior impact are usually based on divergences such as the Kullback–Leibler divergence or distances such as the Wasserstein distance. However, these distances are not directly interpretable and cannot tell us the weight of the impact of a prior distribution. We provide the weight of impact for prior distributions relative to standard normal distributions.

\section{Methodology}
\subsection{Calculation of the Wasserstein distance}

\subsection{Normalisation with prior distances}

\subsection{Power posteriors}

\section{Examples}

\subsection{Normal-normal conjugate case with unknown mean}
Following standard arguments, it is possible to show that the power posterior for the following Bayesian model with dataset $x = (x_1, \ldots, x_n)$
\begin{subequations}
\begin{align}
x_1, \ldots, x_n &\sim \mathcal{N}(m, \sigma^2), \\
m &\sim \mathcal{N}(m_0, \sigma_0^2),
\end{align}
\end{subequations}
is normally distributed
\begin{equation}
m \;|\; x \sim \mathcal{N} \left( \left( \frac{1}{\sigma_0^2} + \frac{\gamma n}{\sigma^2} \right)^{-1} \left(\frac{m_0}{\sigma_0^2} + \frac{\gamma n \bar{x}}{\sigma^2}  \right), \left( \frac{1}{\sigma_0^2} + \frac{\gamma n}{\sigma^2} \right)^{-1} \right),
\end{equation}
where $\bar{x} = (\sum x_i)/ n$ is the sample mean. We remark that, as expected, the prior is recovered when $\gamma = 0$ and the classic normal-normal with unknown mean conjugate result is recovered when $\gamma = 1$. The role of $\gamma$ in this context is to reduce the contribution of each element of the dataset through the likelihood. More specifically for the normal-normal case, the effective data set size is reduced from the standard posterior ($\gamma = 1$) from $n$ to $n \gamma$ in the power posterior. Note however, regardless of the value of $\gamma > 0$, the entire dataset $x$ is still used in the update from prior to the power posterior.

The Wasserstein metric for the $p = 2$ case for two non-degenerate normal measures $\mu_1$ and $\mu_2$ on $\mathbb{R}$ with means $m_1, m_2 \in \mathbb{R}$ and variances $\sigma_1^2, \sigma_2^2 \in \mathbb{R}_{>0} := \left\lbrace x \in \mathbb{R} \; | \; x > 0 \right\rbrace$, respectively, can be defined in closed form as
\begin{equation}
W_{2} (\mu_1, \mu_2)^2 = \left( m_1 - m_2 \right)^2 + \sigma_1^2 + \sigma_2^2 - 2 \bigl( \sigma_2 \sigma_1^2 \sigma_2 \bigr)^{1/2}.
\end{equation}
Consequently in this case there is no need to resort to approximate numerical computations to compute the Wasserstein metric. In closed form the Wasserstein metric between the prior measure $\mu_0$ and the power posterior measure $\mu_\gamma$ is given by
\begin{equation}
W_2(\mu_0, \mu_\gamma)^2 = \sigma_{0}^{2} + \frac{\sigma^{2} \sigma_{0}^{2}}{p} + \frac{\left(\gamma m_{0} n \sigma_{0}^{2} - \gamma n \sigma_{0}^{2} \bar{x}\right)^{2}}{p^{2}} - \frac{2 \sigma \sigma_{0}^{2}}{\sqrt{p}}
\end{equation}
with $p = \gamma n \sigma_{0}^{2} + \sigma^{2}$ and the derivative of the squared Wasserstein distance with respect to $\gamma$ given by
\begin{equation}
\begin{split}
\frac{d}{d\gamma} &[W_2(\mu_0, \mu_\gamma)^2] = \\
p^{-\frac{15}{2}}n \sigma_{0}^{2} &\left(- p^{\frac{11}{2}} \sigma^{2} \sigma_{0}^{2} - 2 p^{\frac{9}{2}} \left(- \gamma n \sigma_{0}^{2} \bar{x} + m_{0} p - m_{0} \sigma^{2}\right) \left(- \gamma n \sigma_{0}^{2} \bar{x} - m_{0} \sigma^{2} + p \bar{x}\right) + p^{6} \sigma \sigma_{0}^{2}\right)
\end{split}
\end{equation}
Both equations were derived using a computer algebra system. Letting $\gamma = 0$ it can be verified that $W_2(\mu_0, \mu_0)^2 = 0$, as expected, and also that $\frac{d}{d\gamma}[W_2(\mu_0, \mu_0)^2] = 0$. 

In the first experiment 

\subsection{Normal-normal conjugate case with unknown variance}

\section{Conclusion}

\end{document}
